\documentclass{beamer}
\mode<presentation>
{
 \usetheme{Marburg}
\usecolortheme{albatross}
\usepackage{beamerouterthemeinfolines}
\usepackage{beamerinnerthemerounded}
}
 
\usepackage{txfonts}
\usepackage[cp1250]{inputenc}
\usepackage[T1]{fontenc}
\usepackage[croatian]{babel}
\usepackage{multimedia}
\usepackage{fancybox}

\title{ \LaTeX - prezentacija}
\subtitle{naslov}
\author{Velimir Baksa}
\institute[PS]{Proizvodno strojarstvo\\
 Veleu\v{c}ili\v{s}te u Vara\v{z}dinu}


\begin{document}


  \begin{frame}
    \titlepage
\transboxin[duration=0.2]
  \end{frame}
 
\begin{frame}
  \frametitle{Tijelo prezentacije}
Osnovni korak u  \LaTeX -u jest definirati tip dokumenta koji  \newline �elimo
kreirati. U slu�aju prezentacije to je \emph{beamer}.
 \emph{Tijelo} svake prezentacije mora zapo�eti sljede�im naredbama:\newline
 \begin{itemize}

     \item
  \alert{$\backslash begin\{frame\}\rightarrow$} po�etak novog slajda\newline
\pause
\item

  \vdots
  \alert{$\backslash end\{frame\}\rightarrow$} kraj slajda.
 \end{itemize}
\transboxout[duration=0.2]
\end{frame}

\begin{frame}
  \frametitle{Naslovna stranica}
Naslovna stranica dokumenta definira se u \emph{preambuli}\newline na sljede�i na�in:\newline
  \alert{\textbackslash title\{Naslov\}}\newline
  \alert{\textbackslash subtitle\{Podnaslov\}}\newline
  \alert{\textbackslash author\{Autor\}}\newline
  \alert{\textbackslash institute\{Fakultet\}}\newline
  \alert{\textbackslash date\{Datum\}}\newline\newline

A poziva se unutar prvog slajda naredbom  \alert{\textbackslash titlepage}
 \transboxin[duration=0.2]
 \end{frame}


\begin{frame}
 \frametitle{Izgled prezentacije}
Temu prezentacije tako�er je potrebno definirati u preambuli\newline
dokumenta. Teme koje oblikuju izgled prezentacije odre�ene su svojim imenima. \newline
Pr: \emph{Boadilla, Bergen, Frankfurt, Hannover, Singapore, Berkley,...} \newline
a pozivaju se na sljede�i na�in:\newline
\alert{\textbackslash usetheme\{ime\}}
 \transboxout[duration=0.2]
\end{frame}
 
\begin{frame}
 \frametitle{Izgled prezentacije}
Nakon �to smo odabrali osnovnu temu mo�emo promijeniti boju prezentacije\newline
pozivaju�i:\newline
  \alert{\textbackslash usecolortheme\{naziv\}}\newline
ili oblikovati elemente unutar slajda:\newline \alert{\textbackslash useinnertheme\{naziv\}}\newline
te oblikovati vanjski izgled slajda: \newline \alert{\textbackslash useoutertheme\{naziv\}}
 \transboxin[duration=0.2]
\end{frame}

\begin{frame}
\frametitle{Primjer}
\begin{itemize}
\item
Naslov i podnaslov pojedinog slajda dobivaju se pozivaju�i:\newline
 \alert{\textbackslash frametitle\{naslov\}}\newline
 \alert{\textbackslash framesubtitle\{podnaslov\}}
\pause
\item
Primjer kako izgleda jedan slajd u  \LaTeX u\newline
 \alert{\textbackslash begin\{frame\}}\newline
  \alert{\textbackslash frametitle\{Moj\_naslov\}}\newline
  \alert{\textbackslash framesubtitle\{Moj\_podnaslov\}}\newline
 Ovo je  \alert{\textbackslash emph\{jedan\}} \textbackslash newline slajd\newline
 na \alert{\textbackslash frame\{kojem\}} mogu napisati moj tekst\newline
 \alert{\textbackslash end\{frame\}}\newline
\pause
\item
Koji nakon procesiranja izgleda....
\end{itemize}
 \transboxin[duration=0.2]
\end{frame}

\begin{frame}
\frametitle{Moj naslov}
\framesubtitle{Moj podnaslov}
Ovo je \emph{jedan}\newline
slajd na \frame{kojem} mogu napisati moj tekst.
 \transboxout[duration=0.2]
\end{frame}

\begin{frame}
\frametitle{Struktura teksta}
Ukoliko tekst koji �elimo kreirati sadr�i liste, nabrajanja\newline
ili dijelove koje ne �elimo otkriti odjednom koristit �emo okoline\newline
\emph{itemize i enumerate}.\newline
Pr;\newline
 \alert{\textbackslash begin\{itemize\}$\left[\langle+ -\rangle\right]$}\newline
 \alert{\textbackslash item Prvi}\newline
 \alert{\textbackslash item Drugi}\newline
 \alert{\textbackslash item Tre�i}\newline
  \alert{\textbackslash end\{itemize\}}
 \transboxin[duration=0.2]
\end{frame}

\begin{frame}
\frametitle{Struktura teksta}
A mogli smo i ovako:\newline
\alert{\textbackslash begin\{itemize\}}\newline
\alert{\textbackslash item}\newline
Prvi\newline
\alert{\textbackslash pause}\newline
\alert{\textbackslash item}\newline
Drugi\newline
\alert{\textbackslash pause}\newline
\alert{\textbackslash item}\newline
Tre�i\newline
\alert{\textbackslash end\{itemize\}}
 \transboxin[duration=0.2]

\end{frame}

\begin{frame}
\frametitle{Struktura teksta}
\begin{itemize}[<+ ->]  
\item Prvi
\item Drugi
\item Tre�i\newline\newline
\item Ili ako ba� �elite......
\end{itemize}
 \transboxout[duration=0.2]
\end{frame}

\begin{frame}
 \frametitle{Struktura teksta}
 \begin{itemize}[<+ -| alert@+>]
\item Prva nagla�ena re�enica
\item Druga nagla�ena re�enica
\item Tre�a nagla�ena re�enica\newline\newline
\end{itemize}
\uncover<4-> {
�to smo postigli sa \alert{\textbackslash begin\{itemize\}$\left[\langle+ -| alert@\rangle +\right]$....}
}
 \transboxin[duration=0.2]
\end{frame}

\begin{frame}
 \frametitle{Struktura teksta}
Enumerate i itemize mogu se koristiti i jedna unutar druge\newline
posebno za liste koje se sastoje od vi�e dijelova:\newline
 \alert{\textbackslash begin\{enumerate\}}\newline
 \alert{\textbackslash item$\langle 1 - \rangle$}  Prva cjelina\newline
 \alert{\textbackslash item$\langle 2 - \rangle$}  Druga cjelina\newline
 \alert{\textbackslash begin\{itemize\}}\newline
 \alert{\textbackslash item} Prvi dio\newline
 \alert{\textbackslash item$\langle 3 - \rangle$}  Drugi dio\newline
 \alert{\textbackslash item$\langle 2 - \rangle$}  Tre�i dio\newline
 \alert{\textbackslash end\{itemize\}}\newline
 \alert{\textbackslash item$\langle 5 - \rangle$}  Tre�a cjelina\newline
 \alert{\textbackslash end\{enumerate\}}
 \transboxin[duration=0.2]
\end{frame}


\begin{frame}
 \frametitle{Struktura teksta}
\begin{enumerate}
\item <1-> Prva cjelina\newline
\item <2-> Druga cjelina\newline
\begin{itemize} 
\item Prvi dio\newline
\item <3->Drugi dio\newline
\item <2->Tre�i dio\newline
\end{itemize}
\item <5-> Tre�a cjelina\newline
\end{enumerate}
 \transboxout[duration=0.2]
\end{frame}

\begin{frame}
 \frametitle{Blokovi}
 Ukoliko �elimo neki tekst posebno naglasiti, pozvati �emo sljede�u \emph{okolinu}: \newline
 \alert{$\textbackslash begin\{block\}\{Naslov\}$}\newline
 $\vdots$ \newline
 \alert{$\backslash end\{block\}$}\newline

koja kreira blokove unutar slajdova na sljede�i na�in....
 \transboxin[duration=0.2]
\end{frame}


\begin{frame}
\frametitle{Blokovi}
\begin{block}{Prvi blok}
\begin{itemize}
 \item Kako izgleda prvi blok
\pause
 \item Unutar kojeg se nalazi tekst
 \end{itemize}
 \end{block}

\begin{block}{Drugi blok}
 \begin{itemize}
 \item Kako izgleda drugi blok
 \end{itemize}
\end{block}
\transboxout[duration=0.2]
\end{frame}


\begin{frame}
\frametitle{Teorem}
\begin{theorem}[Neki teorem]
Ovo je \textbf{neki} teorem
\end{theorem}
\begin{proof}[Neki teorem]
A \textbf{ovo} je dokaz!
\end{proof}
A to smo postigli sa:\newline 
 \alert{$\textbackslash begin\{theorem\}\{Naslov\}$}\newline
 $\vdots$ \newline
 \alert{$\backslash end\{theorem\}$} i\newline \newline
 \alert{$\textbackslash begin\{proof\}$} \newline
 $\vdots$ \newline
 \alert{$\backslash end\{proof\}$}
\transboxin[duration=0.2]
\end{frame}

\begin{frame}
\frametitle{\alert{za kraj...}}
\begin{itemize}
\item
\begin{align}
A+B&=C\uncover<2->{&=D}\uncover<3->{&=E}
\end{align}
\end{itemize}
\begin{itemize}
\item
Mogu koristiti \shadowbox{uokvirenu} kutiju 
\end{itemize}
\transboxout[duration=0.2]
\end{frame}
\begin{frame}
\only<+>{Ova re�enica pojavljuje se samo na ovom slajdu .}
\uncover<+->{   
Ova ju mijenja na sljede�em.} 
\textbf<+>{ \textcolor{blue}{Neke se ponavljaju. }}
\transboxin[duration=0.2]
\end{frame}
\begin{frame}
  \frametitle{\alert{za kraj...}}
Jedan mali klip...
   \movie[width=7cm, height=5cm ]{image}{pijanirus.avi}
\transboxout[duration=0.2]
\end{frame}

\end{document}

